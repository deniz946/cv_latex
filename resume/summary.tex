%-------------------------------------------------------------------------------
%	SECTION TITLE
%-------------------------------------------------------------------------------
\cvsection{Resumen}


%-------------------------------------------------------------------------------
%	CONTENT
%-------------------------------------------------------------------------------
\begin{cvparagraph}

%---------------------------------------------------------
He trabajado anteriormente en una empresa de Marketing Online haciendo SEO y desarrollo de web apps. También he ocupado un puesto en dos inmobiliarias como técnico informático donde creaba y gestionaba sus webs.
 
Actualmente formo parte del equipo de una empresa de desarrollo web en la que nos dedicamos a crear la Intranet de una multinacional de alimentación muy famosa. En mi día a día uso AngularJS, HTML/CSS y GIT.
 
Me considero una persona muy autodidacta y responsable. Una gran parte de las cosas que he aprendido ha sido a través de Internet, cursos y lecturas de las documentaciones propias de las tecnologías.
 
Me gusta mucho viajar y conocer diferentes culturas. Tengo un nivel alto de inglés (técnico) y español, y dominio de búlgaro y turco a nivel nativo.
 
Hace un año he ganado el primer premio del concurso de Hacakthon (hack4good) de Alicante con el proyecto “Internet of Bears” que se caracteriza por el uso de big data, sensores, hardware (raspberry pi) y una aplicación móvil que gestiona toda esa información.
                                                               
Asimismo, estoy creando un proyecto propio desarollado con AngularJS para el frontend y NODEJS en el backend junto a Mongo para la base de datos. Este proyecto es una plataforma web que consiste en intercambio de libros. Por ejemplo, yo tengo un libro que ya lo he leído y quiero leer otros libros. Para ese fin, publico un anuncio en el que ofrezco mi libro a cambio de otro. Todo aquel que esté interesado en dicho libro se podrá poner en contacto conmigo para llevar a cabo el intercambio. Todo gratis, no hay ningún tipo de pago ni dinero de por medio.
 
Los lenguajes/las tecnologías que más domino son HTML5, CSS3, Javascript, AngularJS, NodeJS, Express, Mongoose, Mongo, Laravel, Bootstrap 3 y GIT.
\end{cvparagraph}
